\section{Procesos} \label{process}

\subsection{Estructura de Archivos y Carpetas para la Herramienta}

La herramienta tiene como entradas los cat�logos, 
as� como los datos de ventas, inventarios y precios de competencia.
Estos datos deben de estar bajo cualquier carpeta del sistema, 
la herramienta solo debe conocer la ubicaci�n de la carpeta
durante su funcionamiento.

Con el fin de ser concretos, supondremos que los archivos
est�n guardados en la carpeta /home/black/Documents/chedraui.
Dentro de esta carpeta se deber�n colocar todos los cat�logos,
y se deber�n crear las carpetas inventarios/, ventas/ y preciosCompetencia/. 
Dentro de las �ltimas se deben colocar 
los archivos de inventarios, ventas y precios de competencia respectivamente.
Adicionalmente, la herramienta deposita sus resultados dentro de la carpeta llamada resultados/, 
se recomienda que esta carpeta sea creada antes de correr cualquier funci�n de la herramienta.

Una carpeta con todas las entradas necesarias para la herramienta
tendr�n como ejemplo los siguientes archivos:

\begin{Verbatim}[tabsize=4]
/home/black/Documents/chedraui/
	cat_skus.csv
	cat_excepciones.csv 
	... (todos los archivos de catalogos) 
	cat_matriz.csv
	inventarios/
		Inven_15Ene17.csv
		Inven_14Feb17.csv
		... (todos los archivos de inventarios)
		Inven_22Dic17.csv
	ventas/ 
		ventas201702.csv
		ventas201703.csv
		... (todos los archivos de ventas)
		ventas201751.csv
	preciosCompetencia/
		nielsen201702.csv
		nielsen201703.csv
		otroproveedor201703.csv
		... (todos los archivos de precios)
		nielsen201750.csv
	resultados/
		... (la herramienta deposita resultados)
\end{Verbatim}



\subsection{Sugerencias de Precios}

Una vez que los datos de entrada y los cat�logos est�n colocados en una carpeta con la estructura especificada,
la herramienta puede hacer su trabajo. 
El proceso que se sigue con la herramienta para generar sugerencias de precios generalmente sigue el siguiente esquema:
\begin{enumerate}
	\item Importaci�n de funciones de la herramienta.
	\item Lectura de cat�logos con la funci�n leeCatalogos.
	\item Construcci�n de tablas de precios de la competencia con la funci�n calcCompetencia.
	\item Generaci�n de tabla maestra datos\_<Anio><Semana>.csv con la funci�n calcDatos.
	\item Generaci�n de sugerencias de precios asociadas a la tabla maestra con la funci�n calcPrecios.
	\item Cruce con inventario y ventas con la funci�n calcImpactoPrecios, en este paso tambi�n se calculan los impactos de cambio de precios.
\end{enumerate}

Por ejemplo, sup�ngase que los datos son colocados en la carpeta /home/black/Documents/chedraui/
y se desea generar sugerencias de precios con datos de la semana 201751 utilizando el inventario del 9 de Enero del 2018.
Spark debe estar instalado en la m�quina proporcionada por BlackTrust, 
su consola se debe iniciar junto con el archivo jar de la herramienta,
\begin{verbatim}
$ start-spark
$ spark-shell --jars herramienta-chedraui_2.11-1.0.jar \ 
  --driver-memory 8g
\end{verbatim}
es posible que el nombre del archivo jar sea ligeramente distinto 
y esta l�nea se debe ajustar seg�n ese nombre.

En la pr�ctica se recomienda partir el proceso en dos, 
la generaci�n de la tabla maestra y luego el c�lculo de impactos.
Para la primera parte del proceso, los comandos a seguir en la consola de Spark son: 
\begin{lstlisting}[breaklines, language=Scala]
// Importar funciones
import org.btrust.chedrauiHerramienta.Catalogues.leeCatalogos
import org.btrust.chedrauiHerramienta.Calc.{calcCompetencia,calcDatos,calcPrecios,calcImpactoPrecios}
spark.conf.set("spark.sql.autoBroadcastJoinThreshold",-1)

// Carpeta de datos
val rootDir = "/home/black/Documents/chedraui"

// Agno, semana, inventario
val myYear = 2017
val myWeek = 51
val inven  = "Inven_09Ene18.csv"

// Leer catalogos
val cats  = leeCatalogos(rootDir,matriz="matriz_09Ene2018.csv")

// Generar tablas de competencia
calcCompetencia(myYear,myWeek,cats,rootDir,uri)

// Generar tabla maestra de datos
calcDatos(myYear,myWeek,cats,rootDir,uri=uri,inven=inven)
\end{lstlisting}

En este punto se recomienda cerrar la consola con Control+D y reiniciarla,
el colector de basura se puede saturar despu�s de calcDatos. 
Hasta ahora se han generado los archivos con prefijos ``comp'' y ``datos.''
La segunda parte del proceso es generar la tabla de impactos, 
\begin{lstlisting}[breaklines, language=Scala]
// Importar funciones
import org.btrust.chedrauiHerramienta.Catalogues.leeCatalogos
import org.btrust.chedrauiHerramienta.Calc.{calcCompetencia,calcDatos,calcPrecios,calcImpactoPrecios}
spark.conf.set("spark.sql.autoBroadcastJoinThreshold",-1)

// Carpeta de datos
val rootDir = "/home/black/Documents/chedraui"

// Agno, semana, inventario
val myYear = 2017
val myWeek = 51
val inven  = "Inven_09Ene18.csv"

// Leer catalogos
val cats  = leeCatalogos(rootDir,matriz="matriz_09Ene2018.csv")

// Generar tabla de precios sugeridos
calcPrecios(myYear,myWeek,cats,rootDir,uri)

// Generar tabla de impactos, cruce de precios con inventario
calcImpactoPrecios(myYear,myWeek,cats,inven,rootDir)
\end{lstlisting}
Como en este caso se reinici� la consola, se tienen que declarar las variables
y leer los cat�logos de nuevo. 
Al terminar el proceso, se generan los archivos con prefijo ``precios'' e ``impactos.''



\subsection{Extracci�n de Resultados}

Debido al funcionamiento interno de la herramienta y sus dependencias
los resultados se depositan en el disco duro en particiones (archivos part-).  
Los archivos CSV resultantes son en realidad carpetas que contienen las particiones
y los encabezados de estos archivos.

Junto con la herramienta, se ha incluido una colecci�n de scripts 
que permiten la extracci�n de estas carpetas en un solo archivo consolidado.
Los scripts consisten en la lectura y consolidaci�n de las particiones y encabezados 
de los resultados.  

Los scripts son sc\_listaCSV.bash, sc\_extraeCSV.bash y sc\_extraeTodo.bash.
El archivo README incluido con ellos detalla el uso de estos scripts en la terminal.
